\section{The security of ElGamal}

在本章节中,我们首先介绍 IND-CPA 安全性,之后给出 ElGamal 的 IND-CPA 安全性的等价问题,并说明
直接在模 P 循环群上使用 ElGamal 公钥加密算法将导致 IND-CPA 不安全。最后我们将提出
一种方式使得 ElGamal 公钥加密算法满足 IND-CPA 安全性。

\subsection{IND-CPA 安全性}

除了“直接获得明文”或者“直接获得密钥”的攻击之外,一种更弱的攻击是,虽然攻击者不能直接获得明文或密钥,但能获得一些关于他们的信息,从而可以通过某些统计学方法破译明文。

IND-CPA 是一种对加密算法安全性的描述,满足 IND-CPA 要求的加密算法应该在攻击者选择明文的基础上,产生攻击者无法获得信息的密文。其测试方法包括一个“攻击者”和一个“挑战者”,攻击者将给出两个明文,并从挑战者算出的密文中判断其产生自哪个明文。具体流程如下

\begin{enumerate}
	\item 挑战者基于某些秘密参数 $k$ 生成公私钥 $(P_k, S_k)$,并将公钥 $P_k$ 发给攻击者
	\item 攻击者进行多项式级别的计算
	\item 攻击者向挑战者发送 $M_0, M_1$ 两条明文
	\item 挑战者等概率随机选择两条明文中的一条,记作 $M_b$,将加密后的串 $C_b=E(P_k, M_b)$ 发回给攻击者。
	\item 攻击者进行任意多次计算,并给出对 $b$ 的猜测
\end{enumerate}

因为挑战者收到了密文 $M_b$,与随机猜测 $b$ 相比,他将获得一些优势。记其猜对的概率为 $\frac{1}{2} + \epsilon(k)$,若 $\epsilon(k)$ 并非是一个可以忽略的小量,则该算法不是 IND-CPA 安全的。

\subsection{DDH 问题}

在 ElGamal 的 IND-CPA 证明中,攻击者可以获得的信息包括: $\{P_k, P_r, M_0, M_1, C_b\}$,其中 $P_k = \alpha^k$ 是公钥,$P_r = \alpha^r$ 是加密使用的随机幂。要判断 $C_b$ 来自哪个明文,攻击者可以对两个密文分别计算 $M_x^{-1}C_b$ ,并判断其是否等于 $\alpha^{rk}$,若等于则 $b=x$。我们将这个判断问题简化为下述 DDH 问题:

\begin{theorem}
	给定循环群 $G$ 及其生成元 $\alpha$。给出 $\alpha, \alpha^a, \alpha^b, \alpha^c$,其中 $a, b$ 从 $Z_t$ 中独立随机选择,$c$ 通过下面两种方式产生:$c = ab$ 或从 $Z_t$ 中随机选择。判断 $c$ 是通过哪种方式产生的。
\end{theorem}

不难发现,如果我们可以以一定的概率估计 $\alpha^c$ 是否等于 $\alpha^{ab}$,我们就能对 ElGamal 的 IND-CPA 安全性进行攻击。

\subsection{模 P 循环群上 ElGamal 公钥加密算法的 IND-CPA 攻击}

对模 P 循环群及其生成元 $\alpha$,有下列引理存在:

\begin{lemma}
	$(\alpha^i)^{(P-1)/2} \equiv \pm 1 (mod P)$
\end{lemma}

\begin{proof}
	$((\alpha^i)^{(P-1)/2})^2 = (\alpha^{P-1})^i \equiv 1 (mod P)$。而 $X^2 \equiv 1 (mod P)$ 对质数 $P$ 只有两个解,则他们是 $\pm 1$。
\end{proof}

由于上述引理存在,在判断 $\alpha^c = \alpha^{ab}$ 是否成立时,我们可以通过分别计算 $(\alpha^a)^{(P-1)/2}$, $(\alpha^b)^{(P-1)/2}$, $(\alpha^c)^{(P-1)/2}$ 获得 $a, b, c$ 的奇偶性,从而对结果有所估计。

除了 $(P-1)/2$ 可以用于进行域的收缩外,对任意 $K \ge 2$,$(P-1)/k$ 均可以用于进行数域的收缩。

\subsection{改进的 ElGamal 公钥加密算法}

我们发现,上述对 ElGamal 公钥加密算法的安全性攻击基于 $(P-1)$ 的小质因子成立。首先我们可以尽量避免 $(P-1)$ 的小质因子:选择质数 $q$,另 $P = 2q + 1$ 且也是质数,这样 $(P - 1)$ 就只会有 $2, P$ 两个质因子。

我们接着解决质因子 $2$ 带来的问题。值得注意的是,如果我们要求公钥 $P_k = \alpha^k$ 和随机幂 $P_r = \alpha^r$ 中的 $k$ 和 $r$ 均是偶数,上一章节中提到的 $(P-1)/2$ 幂次攻击将永远得到 $c, ab$ 奇偶性相同的结论,
从而无法进行攻击。这一防范方法的等价做法为:令 $\alpha' = \alpha^2$,则 $\alpha'$ 将生成子群 $\{\alpha^{2 \cdot 1}, \alpha^{2 \cdot 2}, ..., \alpha^{2 \cdot q}$,我们可以基于这个循环子群进行加密:

\begin{enumerate}
	\item 选择大质数 $q$,使得 $P = 2q + 1$ 也是质数
	\item 寻找大质数 $P$ 的生成元 $\alpha$,令 $\alpha' = \alpha^2$
	\item 在 $Z_q$ 中选取私钥 $S_k = k$,公钥为 $\{P, \alpha', P_k = (\alpha')^k\}$
\end{enumerate}